\chapter{Conclusões}
\label{c:conclusoes}

O desenvolvimento do presente projeto possibilitou uma análise de como um sistema feito a partir de um estudo, pode melhorar o estudo do cenário energético da Universidade Federal de Goiás. Além disso também permitiu que fosse analisada a infraestrutura de medição entregue pela Enel à UFG, observando as suas limitações e os seus benefícios para estudos como troca de contrato de demanda, troca de lâmpadas, políticas de redução de consumo, entre outras. Alguns pontos importantes puderam ser obtidos do desenvolvimento desse trabalho:

\begin{itemize}
    \item Multidisciplinaridade: o envolvimento de diversas áreas do conhecimento, como engenharia elétrica, economia, ciências sociais, automação, engenharia de software, entre outras, é de extrema importância para um bom desenvolvimento de um projeto com uma área de atuação tão ampla como este.
    \item Código Livre: o desenvolvimento de um projeto que irá contemplar vários tipos de análises em estudos futuros deve ter o seu código aberto para visualização e contribuição de outros pesquisadores, permitindo que o seu funcionamento não dependa de uma empresa em específico, como é o caso do sistema proprietário da CCK.
    \item Melhoramento da estrutura: um modelo melhor de estrutura física da rede de medição pode ser estudado, de forma a viabilizar uma comunicação mais aberta entre os sensores e os inúmeros meios de interface que poderão existir caso esses dados sejam mais livres. Para o desenvolvimento deste projeto foi necessário inicialmente criar uma maneira de tornar os dados de medição que eram restritos acessíveis livremente, através de um banco de dados, mas se os sensores não fossem de arquitetura fechada, o mesmo poderia ocorrer sem a necessidade de criação desse sistema.
\end{itemize}

Os objetivos iniciais de desenvolver três sistemas responsáveis por obter os dados das medições, obter os dados históricos da concessionária através das faturas, armazenar esses dados, criar uma interface para visualização dos dados de medição, e a implantação de todo esse sistema, disponibilizando-o para uso da instituição, foram atingidos.