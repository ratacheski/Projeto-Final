\chapter{Contextualizando o Cenário Energético Brasileiro}
\label{c:contextualizando_o_cenario_energetico_brasileiro}
% ---

Como mostrado anteriormente, o projeto a ser desenvolvido está totalmente inserido
na abordagem de energia e consumo energético. Por isso é muito importante apresentar de maneira resumida o contexto histórico que estimula cada vez mais o estudo de soluções para possibilitar um consumo mais consciente dos recursos energéticos. 

% ---
\section{Aumento do Consumo de Energia}
% ---
O consumo de energia é um dos principais índices de
desenvolvimento econômico e qualidade de vida
da sociedade. Através dele podemos analisar tanto o ritmo de atividade
do comércio e das indústrias, quanto a
disponibilidade da população em adquirir bens e serviços
mais avançados tecnologicamente, como eletrônicos, automóveis, eletrodomésticos e serviços que necessitam do consumo energético para seu funcionamento.

Além do desenvolvimento econômico, outro fator ao qual o consumo de energia está diretamente relacionado é o crescimento populacional – indicador obtido pelo cálculo diferença entre as taxas de natalidade e mortalidade associado à medição de fluxos migratórios. No Brasil, no período de 2000 a 2005, o crescimento populacional teve uma tendência de queda relativa, registrando variação média anual de 1,46\%, segundo relata o estudo Análise Retrospectiva constante do Plano Nacional de Energia 2030, produzido pela Empresa de Pesquisa Energética.

Ainda assim, a tendência do consumo de energia no período foi de crescimento: 13,93\%, mostrando que mesmo que com uma queda do crescimento populacional a influência do crescimento econômico do cenário analisado se mostrou como mais influente sobre o consumo energético populacional. O Produto Interno Bruto do país, no mesmo período, registrou um
crescimento acumulado de 14,72\%, conforme dados do Ipea.

Analisando um cenário mais recente pode-se observar também aumento no consumo de energia elétrica no país, que totalizou 463.948 gigawatts-hora (GWh) em 2017, correspondendo a um crescimento de 0,8\%, segundo levantamento da Empresa de Pesquisa Energética (EPE). Somente em dezembro, o consumo foi de 39.288 GWh, alta de 1,7\% em relação ao verificado no mesmo período do ano anterior. \cite{atlasenergetico}


O consumo no mercado cativo (atendido pelas distribuidoras) teve queda de 5,6\% em 2017 e de 3\% em dezembro, influenciada pela migração de consumidores para o mercado livre, que cresceu 18,4\% e 13,7\%, respectivamente.

Dentre as classes de consumo, destaque para o segmento industrial, que cresceu 1,3\% no ano de 2017, alcançando 165.883 GWh, após duas quedas consecutivas nos anos anteriores, reflexo da melhora no cenário econômico, o que reforça a dependência direta entre o consumo e a economia.

% ---
\section{Utilização de Fontes Renováveis}
% ---

As energias renováveis se tornam cada vez mais atraentes como alternativa de micro geração distribuída uma vez que houve redução do preço de células fotovoltaicas e aero geradores, já que atualmente aliado ao poder de compra do consumidor está o alto valor de conta de energia elétrica pago pelo brasileiro, considerada uma das contas mais elevadas no mundo. \cite{barbosa2013geraccao}

Fazendo uma análise inicial, as fontes renováveis, aparentemente, possuem um custo final mais elevado do que o sistema convencional centralizado de fornecimento energético. Entretanto o processo como um todo de produção da energia promove uma consequente redução
em seu valor total quando todos os processos necessários são contabilizados.

Os recursos fósseis necessitam de extração, transporte para as refinarias onde são preparados para a queima e, após a geração de eletricidade, esta deve ser transmitida através de linhas de alta tensão para o consumidor, enquanto que os resíduos devem ser eliminados. A utilização de máquinas rotativas, tais como turbina e gerador, necessitam de uma extensa rotina de manutenção , devido ao desgaste natural das peças móveis, além de gerar poluição sonora durante o seu funcionamento. Por outro lado, a energia solar não necessita de um pesado e caro processo de extração, não demanda refinamento e nem transporte para o local da consumo, devido o mesmo ser o qual é próximo ao local de geração, evitando assim os custos com a transmissão em alta tensão. Utiliza células solares, responsáveis pela geração de energia, e um inversor para transformar a tensão gerada em corrente contínua para os valores nominais dos aparelhos de consumo em corrente alternada. Este processo é mais simples, sem emissão de gases poluentes ou ruídos e com uma pequena dependência de manutenções periódicas.\cite{galdino2000contexto}

Os custos envolvendo todas estas etapas necessárias para a geração de
energia devem ser computados no momento em que se compara a energia solar
com as outras fontes. Devido à sua simplicidade, esta forma renovável de obter
eletricidade possui vantagens econômicas. 