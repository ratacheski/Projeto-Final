\chapter{Sistema Sincronizador de Dados}
\label{c:sistema_sincronizador_de_dados}
% ---
Descrita a estrutura de banco de dados do projeto, bem como a infraestrutura da rede \textit{Modbus} de medição da UFG, pode-se entender agora o modo de funcionamento do Sistema Responsável por alimentar o banco com as informações capturadas dos medidores. Para isso foi desenvolvido uma aplicação na linguagem \textit{node js}, que é acionada pelo \textit{crontabs} de um servidor Linux com o acionamento a cada minuto.

O código do \textit{SIDE Synchronizer} é composto basicamente de dois momentos, como mostrado no algoritmo \ref{listing:sidesynchronizer}:

\begin{enumerate}
    \item O sistema acessa o Mestre da rede Modbus e requisita a lista de escravos cadastrados, atualizando o banco de dados com as alterações do mesmo.
    \item O sistema utiliza essa lista de medidores para requisitar as últimas medições feitas em cada medidor a partir da data da última leitura de cada um.
\end{enumerate}

Essas duas etapas serão melhor descritas nas sessões \ref{sec:sincronizacao-medidores} e \ref{sec:sincronizacao-medicoes}, porém, por hora, pode-se analisar o código abaixo da classe principal do \textit{SIDE Synchronizer}.

\begin{listing}[ht]
\caption{Código Principal \textit{SIDE Synchronizer}}
\inputminted[frame=lines, 
    framesep=5mm, fontsize=\footnotesize, linenos=true, label={sidesynchronizer.js}]{js}{codigos/sidesynchronizer.js}
\label{listing:sidesynchronizer}
\end{listing}

Inicialmente, na linha 4 inicializa-se uma lista de medidores que receberá a listagem do método medidores.sincronizaMedidoresCCK() na linha 6. Após o retorno do método e a população do \textit{array}, faz-se a iteração por ele onde a cada iteração é executado o método medicoes.sincronizaMedicoesCCK(medidor) passando o medidor como parâmetro.

\newpage
Após a iteração, na linha 10 se encerra o processo, que em média leva 1,2 segundos para executar na atual situação da planta de medidores da UFG. após 1 minuto o cron do \textit{Linux} chama esse processo novamente, mantendo sempre assim a base atualizada com as medições da rede \textit{Modbus}.

A seguir serão detalhados os dois métodos que fazem a sincronização dos medidores e suas respectivas medições.

\section{Sincronização de Medidores}
\label{sec:sincronizacao-medidores}

Descrito o funcionamento Geral da aplicação, pode-se analisar o código resumido abaixo, da classe responsável pela sincronização inicial dos medidores, como descrito no algoritmo \ref{listing:medidores}.

\begin{listing}[ht]
\caption{Código Resumido da Classe de Medidores}
\inputminted[
    frame=lines, 
    framesep=5mm, 
    fontsize=\footnotesize, 
    linenos=true, 
    label={medidores.js}, 
    firstline=26, 
    lastline=58
    ]{js}{codigos/medidores.js}
\label{listing:medidores}
\end{listing}

O método sincronizaMedidoresCCK faz inicialmente uma listagem de todos os medidores cadastrados na rede e os armazena na variável medidoresCCK, onde na linha 38 inicializa-se uma lista de retornos da consulta que será populada com os medidores cadastrados na rede \textit{Modbus}. Na linha 40 é feita uma requisição http padrão na  Após o retorno do método e a população do \textit{array}, faz-se a iteração por ele onde a cada iteração é executado o método medicoes.sincronizaMedicoesCCK(medidor) passando o medidor como parâmetro. 

Após a listagem, é feita no servidor CCK uma outra validação, onde ma linha 29 após passar os medidores se obtém para cada um a data da última e da primeira leitura que o dispositivo fez, atualizando assim mais uma vez a lista medidoresCCK. 

Após o preenchimento é buscado no banco de dados os medidores já registrados e feito o cadastro caso não existam na base, e atualização da denominação, e datas de primeira e ultima leitura caso já existam na base. Ao final é retornado a lista de medidores atualizada.

\section{Sincronização de Medições}
\label{sec:sincronizacao-medicoes}

Após o retorno da sincronização dos medidores fornecer a lista atualizada de medidores, pode-se iterar essa lista chamando-se, para cada medidor, a sincronização de medições exibida no algoritmo \ref{listing:medicoes}.

\begin{listing}[ht]
\caption{Código Resumido da Classe de Medições}
\inputminted[
    frame=lines, 
    framesep=5mm, 
    fontsize=\footnotesize, 
    linenos=true, 
    label={medicoes.js}, 
    firstline=19, 
    lastline=32
    ]{js}{codigos/medicoes.js}
\label{listing:medicoes}
\end{listing}

\newpage

Inicialmente, na linha 21 é feita uma busca no banco pela última medição do medidor, e posteriormente é feita a seguinte validação: se a data de medição da última leitura não for nula é feita uma busca das últimas leituras existentes no servidor a partir daquela data, caso contrário, entende-se que não existem leituras na tabela de medições para aquele medidor e é feita uma busca por todas as leituras daquele escravo no servidor. 

Após a busca das medições caso ela apresente retorno, as medições são inseridas no banco para este medidor. Este processo é iterado para todos os medidores existentes e após isso a execução do sincronizador é finalizada.